\section{Java Implementation}

We implemented a Java threads version of the framework.
This is a pure Java implementation that fully supports all aspects of the framework.
All grids are written in parallel which is a improvement over the C implementation.

\par
Java's executor service is utilized for parallelization.
A main loop runs using the apache commons CLI interface to read inputs and settings.
The main loop then intelligently divides the work and submits each thread to be executed by the executor service.
The executor services will preform the computations on the maximum amount of cores available.

\par
The NIO (Non-Blocking Input/Output) package is used for concurrent writting and reading of grids.
Essentially, a shared object is passed to each process that gives conccurent read access to the grid.
A writer object is used to queue writes to output files.
This queue is able to write to the output file in a non sequential manner by growing the grid file size in-situ.
This is enabled by the NIO FileChannel class by passing the bytes as a stream to the files or processes.

\par
This implementation uses Java interfaces to define the API.
There are 2 main classes defined in the API.
A Generator and a Collector class.
The generator class is implemented to read a grid and calculate a correponding statistic grid.
Generators need to be thread safe and have all statistics submitted to a collector.
A collector class is implemented to handle generated statistics.
Collector's pipe statistics to a generated file.
This piping needs to be implemented in a non-sequential manner.

\par
There are three modules implemented for the Java framework.
There is a mean, standard deviation, and plane fit module implemented.
The plane fit statistic is implemented using the Plane class from the apache commons math package.
A plane fit can be a very computationally exspensive.
The apache commons plane class does a partial fit that is accurate enough and very fast.
This makes it ideal for fitting accurate planes in a efficent amount of time.

\par
This implementation offers a helpful CLI interface that makes running streamlined.
Allowing a user to specify settings, input, and output grids without recompiling.
This makes the Java implementation the most convient version to use.
Currently, the Java implementation is being used to generate grids for the project.
