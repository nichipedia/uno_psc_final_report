\section{Conclusion}

\par
In this project we developed a parallel modular framework in which we created three separate implementations of statistical feature generation applications.  
By separating the parallel code from the calculation code this gave us the flexibility to change and add new computation modules without editing the parallel logic.  
This solved the issues of having to be very fluid when producing training data for experimental bathymetric prediction algorithms.  
Multiple implementations allowed for computation performance evaluation for different types of statistical features. 
In addition we were able to leverage individual computing paradigms to find which provided the best support for the type of computations we needed to perform.  
Our initial results provided evidence that parallelization drastically improved performance when running computations on extremely large input data grids.  
We also observed that shared memory parallelization seemed to perform better than distributed memory parallelization when computing statistical calculations on large gridded data sets.  
Based on the work performed on this project, initial statistical feature sets have been provided to the geophysicists for bathymetric prediction model training and evaluation. 