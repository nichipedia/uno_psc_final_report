\section{OpenMP Implementation}

We implemented a OpenMP C based version of the framework.
This is a pure C based implementation.
It offers grid buffering, and extremely quick computation.
This is inherit to C's low level execution.

\par
OpenMP's for pragma is used for parallelization.
A public outer for loop is used to split up grid processing across a number of processes.
A private inner for loop is then used to garuntee each process preforms computations on each column.

\par
A shared pointer is used for shared reading across processes.
There are two modes implemented for reading from the grids.
The grids can be read completely into memory which offers very quick reads.
They can also be buffered from files.
This allows for larger grids to be computed on systems with less memory.
Allowing only the required cells to be read as needed for computations.

\par
The interface is defined in C header files.
Succiently, the header files define functions so that the implementation can be abstracted.
Each module needs to implement a "genStat" function and be compiled with the framework to be used.
The same is used for the file writting modules. 
This abstraction allows for modules to be implemented insitu and then read compiled into the executable.

\par
A mean and standard deviation module are currently implemented for the C library.
These modules rely on C's inherit low level computation speed.
They preform well and generate accurate grids.

\par
Implementing modules in the C framework requires a new file that implements the header files.
This file is then compiled into a executable and ran with arguements specifying cores, input, and output grids.