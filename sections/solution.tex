\section{Solution}
For this project we developed a parallel framework to abstract out the individual statistical and mathematical computation components from the parallel implementations. 
It was important do develop this framework due to the nature of the problem we are trying to solve. 
Again when training bathymetric prediction algorithms we don’t know up front what the ideal inputs are for optimal prediction. 
Therefore it is possible that we have to change statistical and mathematical computation regularly to produce improved results. 
Having computation components abstracted from the parallel implementation allows for rapid development and deployment of new statistical and mathematical computations. 
This allows for more time to train prediction models and evaluate results.

\par
Our solution is comprised of three different implementations, one using OpenMP, one using MPI and one using Java and Java threads. 
It is our hypothesis that each programming paradigm would have strengths and weaknesses when running different types of statistical computations. 
In order to find the most efficient use of resources we would compare each implementation against a set of statistical computations to see which performed best. 
We used a 30 second resolution bathymetry dataset as our baseline input and choose mean, standard deviation and plane fit for the statistical computations.

\par
To address memory consumption with large input datasets, we leveraged dynamic memory access. 
Instead of reading an entire grid of data into memory, each process would access dynamically the memory it needed for computation. 
This approach allowed us to process larger data sets on machines with limited memory. 
When applicable we also leveraged buffered I/O. 
Using buffered I/O would limit the amount of disk access when reading data dynamically. 
An example of this can be found in our java implementation. 
Here we used the BufferdReader and BufferedWriter classes to handle all I/O performed by the computation threads.
